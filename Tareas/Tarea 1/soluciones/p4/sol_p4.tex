\documentclass[11pt]{article}


\usepackage[utf8]{inputenc}
\usepackage{fullpage}
\usepackage{epsfig}
\usepackage{amsmath}
\usepackage{amssymb}
\usepackage{multicol}
\usepackage{color}
\usepackage{hyperref}
\usepackage{xcolor}
\usepackage{dirtree}
\usepackage{fontawesome}
\usepackage{tikz}


\usetikzlibrary{trees}


\newcommand{\comm}[1]{{\bf {\color{red} #1}}}
\newcommand{\Enc}{\textit{Enc}}
\newcommand{\Dec}{\textit{Dec}}


\begin{document}

\begin{center}
  \bf Criptografía y Seguridad Computacional - IIC3253\\
  \bf Tarea 1\\
  \bf Solución pregunta 4
\end{center}


\bigskip

\noindent
Considere el juego $\textit{Hash-Col}(n)$ mostrado en clases para definir la noción resistencia a colisiones. Utilizando este tipo de juegos, defina la noción de resistencia a preimagen para una función de hash $(\textit{Gen}, h)$. Además, demuestre que si $(\textit{Gen}, h)$ es resistente a colisiones, entonces $(\textit{Gen}, h)$ es resistente a preimagen.

\bigskip

\noindent
    {\bf Solución.}
Considere una función de hash $(\textit{Gen},h)$ tal que si $\textit{Gen}(1^n) = s$, entonces $h^s
: \{0,1\}^* \to \{0,1\}^{\ell(n)}$ donde $\ell(n)$ es un polinomio
fijo. Además, suponga que $h$ se puede calcular en tiempo polinomial en el largo de la entrada, vale decir, $h(m)$ se puede calcular en tiempo $O(|m|^c)$ para alguna constante fija $c$. Definimos un juego $\textit{Hash-Pre-Img}(n)$ dado por los siguientes pasos:
\begin{enumerate}
\item El verificador genera $s = \textit{Gen}(1^n)$ y un hash $x \in \{0,1\}^{\ell(n)}$
\item El adversario elige $m \in \{0,1\}^*$ o $m = \bot$
\item El adversario gana el juego si alguna de las siguientes condiciones se cumple:
\begin{itemize}
\item $m \in \{0,1\}^{*}$ y $h^s(m) = x$ 
\item $m = \bot$ y  no existe $m' \in \{0,1\}^*$ tal que $h^s(m') = x$
\end{itemize}
En caso contrario, el adversario pierde.
\end{enumerate}
Además, decimos que $(\textit{Gen},h)$ es resistente a preimagen si
para todo adversario que funciona como un algoritmo aleatorizado de
tiempo polinomial, la función $\Pr(\text{Adversario gane }
\textit{Hash-Pre-Img}(n))$ es despreciable (nótese que esta es una
función de $n$).

Vamos a demostrar que si $(\textit{Gen},h)$ es resistente a
colisiones, entonces $(\textit{Gen},h)$ es resistente a preimagen. De
manera más precisa, vamos a hacer esto considerando el contrapositivo,
vale decir, vamos a mostrar que si $(\textit{Gen},h)$ no es resistente
a preimagen, entonces $(\textit{Gen},h)$ no es resistente a
colisiones.

Suponga que $(\textit{Gen},h)$ no es resistente a preimagen. Entonces
existe un adversario ${\cal A}$ tal que ${\cal A}$ es un algoritmo
aleatorizado de tiempo polinomial y $\Pr({\cal A} \text{ gane }
\textit{Hash-Pre-Img}(n))$ no es una función despreciable. A partir
del algoritmo ${\cal A}$, vamos a definir un algoritmo aleatorizado
${\cal B}$ tal que ${\cal B}$ funciona en tiempo polinomial y
$\Pr({\cal B} \text{ gane } \textit{Hash-Col}(n))$ no es una función
despreciable. Suponga que ${\cal A}$ funciona en tiempo $p(n)$, donde
$p(n)$ es un polinomio fijo. Dado $s = \textit{Gen}(1^n)$, el
algoritmo ${\cal B}$ construye $m' = 0^{p(n)+1}$ (vale decir, $m'$
tiene $p(n)+1$ símbolos $0$), se pone en el papel del verificador en
el juego $\textit{Hash-Pre-Img}(n)$ y define $x = h^s(m')$ (nótese que
$x \in \{0,1\}^{\ell(n)}$). Una vez que el algoritmo ${\cal A}$
responde con un mensaje $m \in \{0,1\}^*$ en el juego
$\textit{Hash-Pre-Img}(n)$, el algoritmo ${\cal B}$ retorna el par de
mensajes $m$, $m'$.

Dado que el mensaje $m'$ es de largo $p(n)+1$, la función de hash $h$
se puede calcular en tiempo polinomial (en el largo de la entrada) y
${\cal A}$ es un algoritmo aleatorizado de tiempo polinomial, se tiene
que ${\cal B}$ es un algoritmo aleatorizado de tiempo
polinomial. Para terminar la demostración solo necesitamos
mostrar que $\Pr({\cal B} \text{ gane } \textit{Hash-Col}(n))$ no es
una función despreciable. Nótese que si ${\cal A}$ gana el juego
$\textit{Hash-Pre-Img}(n)$, entonces ${\cal A}$ genera un mensaje $m
\in \{0,1\}^*$ tal que $h(m) = x$ (ya que $x = h(m')$ con $m' \in
\{0,1\}^*$). Además, el algoritmo ${\cal A}$ ejecuta a lo más $p(n)$
pasos, por lo que $|m| \leq p(n)$ y se puede concluir que $m \neq m'$
ya que $|m'| = p(n)+1$. Así, tenemos que si ${\cal A}$ retorna un
mensaje $m \in \{0,1\}^*$ tal que $h(m) = x$, entonces $m$, $m'$ es
una colisión para la función de hash $(\textit{Gen},h)$ y ${\cal B}$
gana el juego $\textit{Hash-Col}(n)$. En términos de probabilidades,
lo que concluimos es que:
\begin{eqnarray*}
  \Pr({\cal B} \text{ gane } \textit{Hash-Col}(n)) &=& 
  \Pr({\cal A} \text{ gane } \textit{Hash-Pre-Img}(n)).
\end{eqnarray*}
De esta forma, se deduce que $\Pr({\cal B} \text{ gane } \textit{Hash-Col}(n))$ es una función no despreciable, puesto que 
  $\Pr({\cal A} \text{ gane } \textit{Hash-Pre-Img}(n))$ es una función no despreciable. Esto concluye
la demostración de la propiedad.
  
\end{document}

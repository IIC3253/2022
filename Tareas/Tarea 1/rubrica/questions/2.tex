
%!TEX root = ../main/main.tex

Considere un esquema criptográfico $(\textit{Gen}, \textit{Enc}, \textit{Dec})$ definido sobre los espacios $\mathcal{M} = \mathcal{K} = \mathcal{C} = \{0,1\}^n$. Suponga además que $\textit{Gen}$ no permite claves cuyo primer bit sea $0$, y que el resto de las claves son elegidas con distribución uniforme. Demuestre que este esquema no es una pseudo-random permutation con una ronda, si $\frac{3}{4}$ es considerada como una probabilidad significativamente mayor a $\frac{1}{2}$.

\medskip

\paragraph{Corrección.}
Esta pregunta se corrige considerando que se debe diseñar una estrategia para el adversario que le permita ganar en una ronda con una probabilidad mayor o igual a $\frac{3}{4}$.
\begin{itemize}
    \item{[3 puntos]} Solo se entrega una estrategia que permite al adversario ganar con una probabilidad mayor o igual a $\frac{3}{4}$.
    
    \item{[4.5 puntos]} Se entrega una estrategia que permite al adversario ganar con una probabilidad mayor o igual a $\frac{3}{4}$, y se formula de manera correcta la probabilidad de que el adversario gane.

    \item{[6 puntos]} Se entrega una estrategia que permite al adversario ganar con una probabilidad mayor o igual a $\frac{3}{4}$, se formula de manera correcta la probabilidad de que el adversario gane, y se muestra que esta probabilidad es mayor o igual a $\frac{3}{4}$.
\end{itemize}

\medskip










\medskip


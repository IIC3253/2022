%!TEX root = ../main/main.tex
\textbf{Disclaimer.} Los algorítmos que deberá implementar en esta pregunta tienen fines exclusivamente académicos y \textbf{no deben ser utilizados en la práctica}. No es recomendado encriptar repetidas veces utilizando la misma llave cuando se utiliza criptografía asimétrica. La criptografía asimétrica se usa en la práctica para intercambiar llaves simétricas y producir firmas digitales. Al encriptar un mensaje usando criptografía asimétrica (por ejemplo, para intercambiar una llave simétrica) se recomienda siempre utilizar versiones aleatorizadas y estandarizadas como lo propuesto en \href{https://en.wikipedia.org/wiki/PKCS_1}{PKCS \#1}.

\paragraph{Enunciado.} En esta pregunta deberá programar dos clases que interactúan entre ellas para comunicarse utilizando el protocolo RSA visto en clases. Concretamente, deberá escribir un Jupyter notebook de acuerdo a las instrucciones explicadas arriba que contenga:

\begin{itemize}
  \item Una clase que represente a quien recibe los mensajes. Esta clase debe permitir generar un par de llaves, entregar la llave pública, y decriptar mensajes.
  \item Una clase que represente a quien envía los mensajes. Para inicializar un objeto de esta clase se debe entregar como parámetro una llave pública con la que luego se debe poder encriptar mensajes.
\end{itemize}

Las firmas de estas clases se deben ver de la siguiente forma:

\begin{python}
  class RSAReceiver:
    
    def __init__(self, bit_len: int) -> None:
    """
    Arguments:
      bit_len: A lower bound for the number of bits of N,
               the second argument of the public and secret key.
    """

    def get_public_key(self) -> bytearray:
    """
    Returns:
      public_key: Public key expressed as a Python 'bytearray' using the
                  PEM format. This means the public key is divided in:
                  (1) The number of bytes of e (4 bytes)
                  (2) the number e (as many bytes as indicated in (1))
                  (3) The number of bytes of N (4 bytes)
                  (4) the number N (as many bytes as indicated in (3))
    """

    def decrypt(self, ciphertext: bytearray) -> str:
    """
    Arguments:
      ciphertext: The ciphertext to decrypt
    Returns:
      message: The original message
    """
\end{python}

\begin{python}
  class RSASender:
    
    def __init__(self, public_key: bytearray) -> None:
    """
    Arguments:
      public_key: The public key that will be used to encrypt messages
    """

    def encrypt(self, message: str) -> bytearray:
    """
    Arguments:
      message: The plaintext message to encrypt
    Returns:
      ciphertext: The encrypted message
    """
\end{python}

Puede definir funciones adicionales antes de definir las clases. También puede definir otros métodos dentro de estas clases. El nombre de todos los métodos y todas las funciones adicionales debe comenzar con un guión bajo (\texttt{\_}).

\paragraph{Detalles de implementación}
\begin{itemize}
  \item Para pasar un mensaje a bytes necesitamos usar alguna codificación particular. Utilizaremos la codificación \href{https://en.wikipedia.org/wiki/UTF-8}{UTF-8}, lo que significa que para pasar un string \texttt{s} a un objeto de tipo \texttt{bytearray} podemos usar \texttt{bytearray(s, 'utf-8')}.
  \item Para encriptar el bytearray proviniente de un mensaje, lo separaremos en bloques de $n$ bytes, donde $n$ es el mayor múltiplo de 8 tal que $8\cdot n$ es menor que el número de bits de $N$. Por ejemplo, si para representar $N$ se necesitan 2056 bits, entonces $n$ será 256, dado que $256\cdot 8=2048$ y 2048 es el mayor múltiplo de 8 estrictamente menor que 2056.
  \item Para encriptar un bloque $b$ (que es un bytearray de tamaño $n$), primero pasaremos dicho bytearray a un número $a$ usando \texttt{int.from\_bytes(b, 'big')}. Luego encriptaremos dicho número como $c=a^e\mod N$. Finalmente, guardaremos $c$ en un bytearray de largo $n+1$, dado que $c$ podría no caber en $n$ bytes.
  \item La cantidad de bytes del mensaje no tiene por qué dividir al número $n$. En caso de que no lo divida simplemente utilizaremos un último bloque más corto. Si el mensaje tiene $\ell$ bytes entonces el último bloque tendrá $\ell\mod n$ bytes.
  \item Supondremos que el byte 0 nunca vendrá codificado en el string de entrada, lo que resulta útil para decriptar el último bloque.
\end{itemize}
